% \documentclass[12pt, unicode]{beamer}
% \documentclass[dvipdfmx, 12pt]{beamer}  % コンパイルには dvipdfmx を用いる
\documentclass[dvipdfmx]{beamer}  % 最小限?

\usepackage{pxjahyper}  % tableOfContents の文字化け防止
\usepackage{graphicx}  % 各種画像の差し込み用ライブラリ
\usepackage{amsmath, amssymb}  % 標準数式表現の拡張

\renewcommand{\kanjifamilydefault}{\gtdefault}  % 日本語フォントの既定をゴシック体にする

\usetheme{Copenhagen}

% -- >8--  >8--  >8-- >8-- >8-- >8-- >8-- >8-- >8--  >8-- >8-- >8-- %
\title{Beamer Tutorial}
\author[sudachi]{鈴木 大智}
\date[2021/02/25]{\TeX でスライドを作ろう with Beamer}
\institute[日本大学大学院]{日本大学大学院 理工学研究科 数学専攻}
% [..] に省略した表現が書ける


% -- >8--  >8--  >8-- >8-- >8-- >8-- >8-- >8-- >8--  >8-- >8-- >8-- %
\begin{document}

\frame{\maketitle}

% -- >8--  >8--  >8-- >8-- >8-- >8-- >8-- >8-- >8--  >8-- >8-- >8-- %
\begin{frame}{frame title}

  果物について
  
  \begin{itemize}
    \item apple
    \item orange
    \item banana
  \end{itemize}
\end{frame}

% -- >8--  >8--  >8-- >8-- >8-- >8-- >8-- >8-- >8--  >8-- >8-- >8-- %
\begin{frame}{blox 環境}

  \begin{block}{概略}
    \begin{itemize}
      \item block, alertblock, exampleblock 環境がある。
      \item ブロックの領域の装飾は、usetheme で指定したテーマに基づいて施される。
    \end{itemize}
  \end{block}

  \begin{alertblock}{注意}
    これは、alertblock の中身です。
  \end{alertblock}

  \begin{exampleblock}{例}
    これは、exampleblock の中身です。
  \end{exampleblock}

  \begin{alertblock}{注意}
    block には、タイトルがないとコンパイルエラーになります。
  \end{alertblock}

\end{frame}

% -- >8--  >8--  >8-- >8-- >8-- >8-- >8-- >8-- >8--  >8-- >8-- >8-- %
\begin{frame}{数式も試してみる}
  $y = x^2 + 3x -5$

  \begin{eqnarray*}
    y &=& x^2 + 3x -5 \\
    z &=& \log(x)
  \end{eqnarray*}

  この和 ~ $ {\displaystyle
    \sum_{i=0}^{N} \frac{1}{n^2}
  } $ ~ を考えたい。あ、考え終わっちゃった。

  \begin{block}{ イプシロン-N 論法 }
    $ {\displaystyle
      {}^{\forall} \varepsilon > 0, ~ {}^{\exists} N \in \mathbb{N}, ~ n \geq N ~
      s.t. ~ | x - x_n | < \varepsilon
    } $
  \end{block}

  \begin{exampleblock}{例}
    \begin{displaymath}
      \lim_{n \rightarrow \infty} \frac{1}{2^n} = 0
    \end{displaymath}
  \end{exampleblock}

\end{frame}

% -- >8--  >8--  >8-- >8-- >8-- >8-- >8-- >8-- >8--  >8-- >8-- >8-- %
\begin{frame}{ハイライト}
  強調には2種類ある。

  \begin{description}
    \item[structure]: ここを \structure{強調} します。
    \item[alert]: \alert{重要} なところを強調します。
  \end{description}
\end{frame}

% -- >8--  >8--  >8-- >8-- >8-- >8-- >8-- >8-- >8--  >8-- >8-- >8-- %
\begin{frame}{columns を利用した配置}
  \begin{block}{top block}
    上段のブロック
  \end{block}

  \begin{columns}[c]  % 中央を合わせる
  % \begin{columns}[b]  % 下を合わせる

    % column1
    \begin{column}{0.3\textwidth}  % 横幅の30%
      \includegraphics[width=\columnwidth]{./media/kenta2.jpg}
    \end{column}

    % column2
    \begin{column}{0.65\textwidth}  % 横幅の65%
      \begin{block}{図の説明}
        うちの家族で飼っている犬です。\\
        名前を「けんた」と言います。
      \end{block}
    \end{column}

    % \textwidthを使用する事がポイント。
    %   これで、表示しているスライドの横幅の長さを取得できるので、コラムの幅の指定が楽になるそう。
    % またコラム中で図を挿入する際は \columnwidth でコラムの幅を取得すると捗るらしい。

  \end{columns}

  \begin{block}{bottom block}
    下段のブロック
  \end{block}

\end{frame}

% -- >8--  >8--  >8-- >8-- >8-- >8-- >8-- >8-- >8--  >8-- >8-- >8-- %
\begin{frame}{pause を用いたアニメーション}
  % \pause をブロックの頭につけることで、ページ送り毎に
  %   ブロック1 -> 2 -> 3 という風に表示される。

  \begin{block}{top block}
    一番上のブロックです。
  \end{block}

  \pause
  \begin{block}{second block}
    二番目のブロックです。
  \end{block}

  \pause
  \begin{block}{third block}
    3番目のブロックです。
  \end{block}

\end{frame}

% -- >8--  >8--  >8-- >8-- >8-- >8-- >8-- >8-- >8--  >8-- >8-- >8-- %
\begin{frame}{pause と uncover}
  % タイミングの指定:
  %   <i> : i番目だけ
  %   <-i>: i番目まで
  %   <i->: i番目以降
  % を表す。
  %   pause は単純に、「次のスライドで出す」という隠し方で、
  %   uncover はより柔軟に設定できる、という印象です。

  \begin{columns}[c]  % 中央をあわせる

    \begin{column}{0.3\textwidth}
      \includegraphics[width=\columnwidth]{./media/kenta2.jpg}
    \end{column}

    \begin{column}{0.65\textwidth} % 横幅の65%
      \begin{block}{コメントブロック}
        \begin{itemize}
          \item<1-> タイミング1「以降」出現するコメント
          \item<2-> タイミング2「以降」出現するコメント
          \item<-3> タイミング3「まで」出現する (4で消える) コメント
        \end{itemize}
      \end{block}
    \end{column}

  \end{columns}

  \uncover<4>{
    \begin{block}{ダジャレ}
      ふとんがふっとんだ!!
    \end{block}
  }

  \uncover<5>{
    \begin{alertblock}{注意}
      つまらないしゃれはやめなしゃれ。
    \end{alertblock}
  }

\end{frame}

% -- >8--  >8--  >8-- >8-- >8-- >8-- >8-- >8-- >8--  >8-- >8-- >8-- %
\frame{\centering \Large Thank you for your attention!!}

\end{document}
