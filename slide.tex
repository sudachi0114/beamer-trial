% \documentclass[12pt, unicode]{beamer}
\documentclass[dvipdfmx, 12pt]{beamer}  % コンパイルには dvipdfmx を用いる

\usetheme{Copenhagen}

\renewcommand{\kanjifamilydefault}{\gtdefault}  % 既定をゴシック体にする

\title{Beamer Tutorial}
\author{鈴木 大智}
\date[2021/02/23]{\TeX でスライドを作ろう with Beamer}
\institute[日本大学大学院]{日本大学大学院 理工学研究科 数学専攻}
% [..] に省略した表現が書ける

\begin{document}

\frame{\maketitle}

\begin{frame}{frame title}

  果物について
  
  \begin{itemize}
    \item apple
    \item orange
    \item banana
  \end{itemize}
\end{frame}

\begin{frame}{blox 環境}

  \begin{block}{概略}
    \begin{itemize}
      \item block, alertblock, exampleblock 環境がある。
      \item ブロックの領域の装飾は、usetheme で指定したテーマに基づいて施される。
    \end{itemize}
  \end{block}

  \begin{alertblock}{注意}
    これは、alertblock の中身です。
  \end{alertblock}

  \begin{exampleblock}{例}
    これは、exampleblock の中身です。
  \end{exampleblock}

  \begin{alertblock}{注意}
    block には、タイトルがないとコンパイルエラーになります。
  \end{alertblock}

\end{frame}

\frame{\centering \Large Thank you for your attention!!}

\end{document}
